% Options for packages loaded elsewhere
\PassOptionsToPackage{unicode}{hyperref}
\PassOptionsToPackage{hyphens}{url}
%
\documentclass[
]{article}
\usepackage{lmodern}
\usepackage{amsmath}
\usepackage{ifxetex,ifluatex}
\ifnum 0\ifxetex 1\fi\ifluatex 1\fi=0 % if pdftex
  \usepackage[T1]{fontenc}
  \usepackage[utf8]{inputenc}
  \usepackage{textcomp} % provide euro and other symbols
  \usepackage{amssymb}
\else % if luatex or xetex
  \usepackage{unicode-math}
  \defaultfontfeatures{Scale=MatchLowercase}
  \defaultfontfeatures[\rmfamily]{Ligatures=TeX,Scale=1}
\fi
% Use upquote if available, for straight quotes in verbatim environments
\IfFileExists{upquote.sty}{\usepackage{upquote}}{}
\IfFileExists{microtype.sty}{% use microtype if available
  \usepackage[]{microtype}
  \UseMicrotypeSet[protrusion]{basicmath} % disable protrusion for tt fonts
}{}
\makeatletter
\@ifundefined{KOMAClassName}{% if non-KOMA class
  \IfFileExists{parskip.sty}{%
    \usepackage{parskip}
  }{% else
    \setlength{\parindent}{0pt}
    \setlength{\parskip}{6pt plus 2pt minus 1pt}}
}{% if KOMA class
  \KOMAoptions{parskip=half}}
\makeatother
\usepackage{xcolor}
\IfFileExists{xurl.sty}{\usepackage{xurl}}{} % add URL line breaks if available
\IfFileExists{bookmark.sty}{\usepackage{bookmark}}{\usepackage{hyperref}}
\hypersetup{
  pdftitle={VectorBiTE 2019 Training 2019},
  pdfauthor={Your Name Here},
  hidelinks,
  pdfcreator={LaTeX via pandoc}}
\urlstyle{same} % disable monospaced font for URLs
\usepackage[margin=1in]{geometry}
\usepackage{color}
\usepackage{fancyvrb}
\newcommand{\VerbBar}{|}
\newcommand{\VERB}{\Verb[commandchars=\\\{\}]}
\DefineVerbatimEnvironment{Highlighting}{Verbatim}{commandchars=\\\{\}}
% Add ',fontsize=\small' for more characters per line
\usepackage{framed}
\definecolor{shadecolor}{RGB}{248,248,248}
\newenvironment{Shaded}{\begin{snugshade}}{\end{snugshade}}
\newcommand{\AlertTok}[1]{\textcolor[rgb]{0.94,0.16,0.16}{#1}}
\newcommand{\AnnotationTok}[1]{\textcolor[rgb]{0.56,0.35,0.01}{\textbf{\textit{#1}}}}
\newcommand{\AttributeTok}[1]{\textcolor[rgb]{0.77,0.63,0.00}{#1}}
\newcommand{\BaseNTok}[1]{\textcolor[rgb]{0.00,0.00,0.81}{#1}}
\newcommand{\BuiltInTok}[1]{#1}
\newcommand{\CharTok}[1]{\textcolor[rgb]{0.31,0.60,0.02}{#1}}
\newcommand{\CommentTok}[1]{\textcolor[rgb]{0.56,0.35,0.01}{\textit{#1}}}
\newcommand{\CommentVarTok}[1]{\textcolor[rgb]{0.56,0.35,0.01}{\textbf{\textit{#1}}}}
\newcommand{\ConstantTok}[1]{\textcolor[rgb]{0.00,0.00,0.00}{#1}}
\newcommand{\ControlFlowTok}[1]{\textcolor[rgb]{0.13,0.29,0.53}{\textbf{#1}}}
\newcommand{\DataTypeTok}[1]{\textcolor[rgb]{0.13,0.29,0.53}{#1}}
\newcommand{\DecValTok}[1]{\textcolor[rgb]{0.00,0.00,0.81}{#1}}
\newcommand{\DocumentationTok}[1]{\textcolor[rgb]{0.56,0.35,0.01}{\textbf{\textit{#1}}}}
\newcommand{\ErrorTok}[1]{\textcolor[rgb]{0.64,0.00,0.00}{\textbf{#1}}}
\newcommand{\ExtensionTok}[1]{#1}
\newcommand{\FloatTok}[1]{\textcolor[rgb]{0.00,0.00,0.81}{#1}}
\newcommand{\FunctionTok}[1]{\textcolor[rgb]{0.00,0.00,0.00}{#1}}
\newcommand{\ImportTok}[1]{#1}
\newcommand{\InformationTok}[1]{\textcolor[rgb]{0.56,0.35,0.01}{\textbf{\textit{#1}}}}
\newcommand{\KeywordTok}[1]{\textcolor[rgb]{0.13,0.29,0.53}{\textbf{#1}}}
\newcommand{\NormalTok}[1]{#1}
\newcommand{\OperatorTok}[1]{\textcolor[rgb]{0.81,0.36,0.00}{\textbf{#1}}}
\newcommand{\OtherTok}[1]{\textcolor[rgb]{0.56,0.35,0.01}{#1}}
\newcommand{\PreprocessorTok}[1]{\textcolor[rgb]{0.56,0.35,0.01}{\textit{#1}}}
\newcommand{\RegionMarkerTok}[1]{#1}
\newcommand{\SpecialCharTok}[1]{\textcolor[rgb]{0.00,0.00,0.00}{#1}}
\newcommand{\SpecialStringTok}[1]{\textcolor[rgb]{0.31,0.60,0.02}{#1}}
\newcommand{\StringTok}[1]{\textcolor[rgb]{0.31,0.60,0.02}{#1}}
\newcommand{\VariableTok}[1]{\textcolor[rgb]{0.00,0.00,0.00}{#1}}
\newcommand{\VerbatimStringTok}[1]{\textcolor[rgb]{0.31,0.60,0.02}{#1}}
\newcommand{\WarningTok}[1]{\textcolor[rgb]{0.56,0.35,0.01}{\textbf{\textit{#1}}}}
\usepackage{graphicx}
\makeatletter
\def\maxwidth{\ifdim\Gin@nat@width>\linewidth\linewidth\else\Gin@nat@width\fi}
\def\maxheight{\ifdim\Gin@nat@height>\textheight\textheight\else\Gin@nat@height\fi}
\makeatother
% Scale images if necessary, so that they will not overflow the page
% margins by default, and it is still possible to overwrite the defaults
% using explicit options in \includegraphics[width, height, ...]{}
\setkeys{Gin}{width=\maxwidth,height=\maxheight,keepaspectratio}
% Set default figure placement to htbp
\makeatletter
\def\fps@figure{htbp}
\makeatother
\setlength{\emergencystretch}{3em} % prevent overfull lines
\providecommand{\tightlist}{%
  \setlength{\itemsep}{0pt}\setlength{\parskip}{0pt}}
\setcounter{secnumdepth}{-\maxdimen} % remove section numbering
\ifluatex
  \usepackage{selnolig}  % disable illegal ligatures
\fi

\title{VectorBiTE 2019 Training 2019}
\usepackage{etoolbox}
\makeatletter
\providecommand{\subtitle}[1]{% add subtitle to \maketitle
  \apptocmd{\@title}{\par {\large #1 \par}}{}{}
}
\makeatother
\subtitle{Activity: Method of Moments and Likelihoods}
\author{Your Name Here}
\date{June 2019}

\begin{document}
\maketitle

\hypertarget{the-binomial-distribution}{%
\subsection{The Binomial Distribution}\label{the-binomial-distribution}}

Used to model the number of ``successes'\,' in a set of trials (e.g.,
number of heads when you flip a coin \(N\) times). The pmf is
\begin{align*}
{N \choose x} p^x(1-p)^{N-x}
\end{align*} such that \(\mathrm{E}[x]=Np\). Throughout this lab, you
will assume that your experiment consists of flipping 20 coins, so that
\(N=20\).

You will use the Binomial distribution to practice two methods of
estimating parameters for a probability distribution: method of moments
and maximum likelihood.

\hypertarget{simulating-from-the-binomial-using-r}{%
\subsubsection{Simulating from the Binomial using
R}\label{simulating-from-the-binomial-using-r}}

Take 50 draws from a binomial (using \emph{rbinom}) for each \(p\in\)
0.1, 0.5, 0.8 with \(N=20\).

\begin{Shaded}
\begin{Highlighting}[]
\DocumentationTok{\#\# 50 draws with each p }
\NormalTok{pp}\OtherTok{\textless{}{-}}\FunctionTok{c}\NormalTok{(}\FloatTok{0.1}\NormalTok{, }\FloatTok{0.5}\NormalTok{, }\FloatTok{0.8}\NormalTok{)}
\NormalTok{N}\OtherTok{\textless{}{-}}\DecValTok{20}
\NormalTok{reps}\OtherTok{\textless{}{-}}\DecValTok{50}
\end{Highlighting}
\end{Shaded}

Plot the histograms of these draws together with the density functions.

\begin{Shaded}
\begin{Highlighting}[]
\DocumentationTok{\#\# histograms + density here}
\NormalTok{x}\OtherTok{\textless{}{-}}\FunctionTok{seq}\NormalTok{(}\DecValTok{0}\NormalTok{, }\DecValTok{50}\NormalTok{, }\AttributeTok{by=}\DecValTok{1}\NormalTok{)}
\FunctionTok{par}\NormalTok{(}\AttributeTok{mfrow=}\FunctionTok{c}\NormalTok{(}\DecValTok{1}\NormalTok{,}\DecValTok{3}\NormalTok{), }\AttributeTok{bty=}\StringTok{"n"}\NormalTok{)}
\end{Highlighting}
\end{Shaded}

\textbf{\emph{Q1: Do the histograms look like the distributions for all
3 values of \(p\)? If not, what do you think is going on?}}

You'll notice that for \(p=0.1\) the histogram and densities don't look
quite the same -- the \emph{hist()} function is lumping together the
zeros and ones which makes it look off. This is typical for
distributions that are truncated.

\hypertarget{method-of-moments-mom-estimators}{%
\subsubsection{Method of Moments (MoM)
Estimators}\label{method-of-moments-mom-estimators}}

To obtain a method of moments estimator, we equate the theoretical
moments (which will be a function of the parameters of the distribution)
with the corresponding sample moments, and solve for the parameters in
order to obtain an estimate. For the binomial distribution, there is
only one parameter, \(p\).

\textbf{\emph{Q2: Given the analytic expected value, above, and assuming
that the sample mean is \(m\) (the mean number of observed heads across
replicates), what is the MoM estimator for \(p\)?}}

Now calculate the MoM estimator for each of your 3 sets of simulated
data sets to get the estimates for each of your values of \(p\).

\begin{Shaded}
\begin{Highlighting}[]
\DocumentationTok{\#\# MOM estimators for 3 simulated sets}
\end{Highlighting}
\end{Shaded}

\textbf{\emph{Q3: How good are your estimates for \(p\)? Do you get
something close to the true value?}}

For 1 of your values of \(p\), take 20 draws from the binomial with
\(N=20\) and calculate the MoM. Repeat this 100 times (hint: the
\emph{replicate()} and \emph{lapply} functions may be useful.) Make a
histogram of your estimates, and add a line/point to the plot to
indicate the real value of \(p\) that you used to simulate the data.

\begin{Shaded}
\begin{Highlighting}[]
\DocumentationTok{\#\# MoM estimates, histogram }
\end{Highlighting}
\end{Shaded}

\textbf{\emph{Q4: Is the MoM successfully estimating \(p\)? Does your
histogram for \(p\) look more or less normal? If so, what theorem might
explain why this would be the case?}}

\hypertarget{mle-for-binomial-distribution}{%
\subsubsection{MLE for Binomial
Distribution}\label{mle-for-binomial-distribution}}

\hypertarget{likelihood-and-log-likelihood}{%
\paragraph{Likelihood and Log
Likelihood}\label{likelihood-and-log-likelihood}}

Imagine that you flip a coin \(N\) times, and then repeat the experiment
\(n\) times. Thus, you have data \(x=x_1, x_2, \dots x_n\) that are the
number of times you observed a head in each trial. \(p\) is the
probability of obtaining a head.

\textbf{\emph{Q5: Write down the likelihood and log-likelihood for the
data. Take the derivative of the negative log-likelihood, set this equal
to zero, and find the MLE, \(\hat{p}\).}}

\hypertarget{computing-the-likelihood-and-mle-in-r}{%
\subsubsection{Computing the likelihood and MLE in
R}\label{computing-the-likelihood-and-mle-in-r}}

Simulate some data with \(p=0.25\), \(N=10\), and 10 replicates.
Calculate the negative log-likelihood of your simulated data across a
range of \(p\) (from 0 to 1), and plot them. You may do this by using
the built in functions in R (specifically \emph{dbinom}) or write your
own function. This is called a ``likelihood profile'\,'. Plot your
likelihood profile with a line indicating the true value of \(p\). Add
lines indicating the MLE \(\hat{p}\) and the MoM estimator for \(p\) to
your likelihood profile.

\begin{Shaded}
\begin{Highlighting}[]
\NormalTok{pp}\OtherTok{\textless{}{-}}\NormalTok{.}\DecValTok{25}
\NormalTok{N}\OtherTok{\textless{}{-}}\DecValTok{10}
\NormalTok{reps}\OtherTok{\textless{}{-}}\DecValTok{10}

\DocumentationTok{\#\# Make one set of data}

\DocumentationTok{\#\# the likelihood is always exactly zero}
\DocumentationTok{\#\# at p=0,1, so I skip those values}
\NormalTok{ps}\OtherTok{\textless{}{-}}\FunctionTok{seq}\NormalTok{(}\FloatTok{0.01}\NormalTok{, }\FloatTok{0.99}\NormalTok{, }\AttributeTok{by=}\FloatTok{0.01}\NormalTok{) }

\DocumentationTok{\#\# Likelihood}


\DocumentationTok{\#\# MLE/MoM estimators }

\DocumentationTok{\#\# now plot the negative log likelihood profile}
\end{Highlighting}
\end{Shaded}

\textbf{\emph{Q6: How does your MLE compare to the true value? If you
chose another version of the random seed, do you get the same answer?}}

\end{document}
